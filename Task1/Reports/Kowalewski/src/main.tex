\documentclass{classrep}
\usepackage[utf8]{inputenc}
\frenchspacing

\usepackage{graphicx}
\usepackage[usenames,dvipsnames]{color}
\usepackage[hidelinks]{hyperref}
\usepackage{lmodern}
\usepackage{graphicx}
\usepackage{placeins}
\usepackage{url}
\usepackage{amsmath, amssymb, mathtools}
\usepackage{listings}
\usepackage{fancyhdr, lastpage}

\pagestyle{fancyplain}
\fancyhf{}
\renewcommand{\headrulewidth}{0pt}
\cfoot{\thepage\ / \pageref*{LastPage}}

%--------------------------------------------------------------------------------------%
\studycycle{Informatyka stosowana, studia dzienne, II st.}
\coursesemester{I}

\coursename{Wprowadzenie do Data Science i metod uczenia maszynowego}
\courseyear{2020/2021}

\courseteacher{mgr inż. Rafał Woźniak}
\coursegroup{Wtorek, 13:15}

\author{%
    \studentinfo[239676@edu.p.lodz.pl]{Kamil Kowalewski}{239676}
}

\title{Zadanie 1.: Problem Set 1}

\begin{document}
    \maketitle
    \thispagestyle{fancyplain}

    \section{Wprowadzenie} {
        Bardzo częstym zjawiskiem jest manipulacja danymi tak aby zmylić odbiorce
        i~zmusić go aby myślał tak ja autor danego tesktu czy też przekazu chciał.
        Może być to realizowane w zróżnicowany sposób natomiast poniżej zostanie
        omówiony problem o nazwie \emph{Korelacja a przyczynowość}
        (ang. \emph{Correlation vs Causation}). Polega on na powiązaniu danej tematyki
        z~inną, która bardzo często jest nieistotna. Ten zabieg jest wykorzystywa aby
        prezenter osiągnał swój cel.
    }

    \section{Błędne podejście do analizy statystycznej} {

    }

    \section{Poprawne podejście do analizy statystycznej} {

    }

    \section{Wnioski} {
        Podsumowując wykonane zadanie wnioskuje, że:
        \begin{itemize}
            \item ...

        \end{itemize}
    }

    \begin{thebibliography}{0}
        \bibitem
        {medium_article}
        {https://medium.com/seek-blog/how-to-lie-with-statistics-b671b66399d}
        \bibitem
        {towardsdatascience_article}
        {https://towardsdatascience
        .com/lessons-from-how-to-lie-with-statistics-57060c0d2f19}
    \end{thebibliography}

\end{document}
