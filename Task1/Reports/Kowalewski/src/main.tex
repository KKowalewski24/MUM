\documentclass{classrep}
\usepackage[utf8]{inputenc}
\frenchspacing

\usepackage{graphicx}
\usepackage[usenames,dvipsnames]{color}
\usepackage[hidelinks]{hyperref}
\usepackage{lmodern}
\usepackage{placeins}
\usepackage{amsmath, amssymb, mathtools}
\usepackage{listings}
\usepackage{fancyhdr, lastpage}

\pagestyle{fancyplain}
\fancyhf{}
\renewcommand{\headrulewidth}{0pt}
\cfoot{\thepage\ / \pageref*{LastPage}}

%--------------------------------------------------------------------------------------%
\studycycle{Informatyka stosowana, studia dzienne, II st.}
\coursesemester{I}

\coursename{Wprowadzenie do Data Science i metod uczenia maszynowego}
\courseyear{2020/2021}

\courseteacher{mgr inż. Rafał Woźniak}
\coursegroup{Wtorek, 13:15}

\author{%
    \studentinfo[239676@edu.p.lodz.pl]{Kamil Kowalewski}{239676}
}

\title{Zadanie 1.: Problem Set 1}

\begin{document}
    \maketitle
    \thispagestyle{fancyplain}

    \section{Wprowadzenie} {
        Bardzo częstym zjawiskiem jest manipulacja danymi tak, aby zmylić odbiorcę
        i~zmusić go, aby myślał, tak ja chciał autor danego tekstu czy też przekazu.
        Może być to realizowane w zróżnicowany sposób natomiast poniżej zostanie
        omówiony problem o nazwie \emph{Korelacja a przyczynowość}
        (ang. \emph{Correlation vs Causation}). Polega on na powiązaniu dwóch tematyk
        czy też zjawisk, jakie mają miejsce i są brane pod uwagę w danych badaniach.
        Można hipotetycznie założyć, że mamy dwie zmienne \textit{X} oraz \textit{Y}
        i~wystepuję między nimi korelacja, czyli zależność jednej od drugiej. Zależność
        ta wyraża się w ten sposób, że przykładowo gdy dwie zmienne \textit{X}
        oraz \textit{Y} wzrastają razem, oraz maleją razem. Celem wyjaśnienia tego można
        wyróżnić parę scenariuszy, pierwsze z nich jest to, że zjawisko określone jako
        zmienna \textit{X} powoduje zjawisko określone przez zmienną \textit{Y}.
        Kolejnym przypadkiem jest to, że zjawisko określone przez zmienną \textit{Y}
        może powodować zjawisko określone przez zmienną \textit{X}. Trzecią opcją jest
        to, że istnieję dodatkowa, trzecia zmienna, która wpływa na wcześniej już
        wspomniane zmienne \textit{X} oraz \textit{Y}. Ostatnią z możliwości jest to,
        że zmienne \textit{X} oraz \textit{Y} są totalnie z sobą niezwiązane a autor
        artykułu powiązał je, aby osiągnąć swój cel i zmanipulować przesłaniem a co za
        tym idzie wnioskami, jaki wyciągnie z nich czytelnik.
    }

    \section{} {
        \begin{figure}[!htbp]
            \centering
            \includegraphics[width=\textwidth,keepaspectratio]{img/margarine_divorces.png}
            \caption
            [Wykres spożycia margaryny oraz liczby rozwodów w danych latach]
            {Wykres spożycia margaryny oraz liczby rozwodów w danych latach}
            \label{margarine_divorces}
        \end{figure}
        \FloatBarrier
    }

    \section{Wnioski} {
        Podsumowując można powiedzieć, że:
        \begin{itemize}
            \item ...

        \end{itemize}
    }

    \begin{thebibliography}{0}
        \bibitem
        {medium_article}
        {https://medium.com/seek-blog/how-to-lie-with-statistics-b671b66399d}
        \bibitem
        {towardsdatascience_article}
        {https://towardsdatascience.com/lessons-from-how-to-lie-with-statistics-57060c0d2f19}
    \end{thebibliography}

\end{document}
