\documentclass{classrep}
\usepackage[utf8]{inputenc}
\frenchspacing

\usepackage{graphicx}
\usepackage[usenames,dvipsnames]{color}
\usepackage[hidelinks]{hyperref}
\usepackage{lmodern}
\usepackage{graphicx}
\usepackage{placeins}
\usepackage{url}
\usepackage{amsmath, amssymb, mathtools}
\usepackage{listings}
\usepackage{fancyhdr, lastpage}

\pagestyle{fancyplain}
\fancyhf{}
\renewcommand{\headrulewidth}{0pt}
\cfoot{\thepage\ / \pageref*{LastPage}}

%--------------------------------------------------------------------------------------%
\studycycle{Informatyka stosowana, studia dzienne, II st.}
\coursesemester{I}

\coursename{Wprowadzenie do Data Science i metod uczenia maszynowego}
\courseyear{2020/2021}

\courseteacher{mgr inż. Rafał Woźniak}
\coursegroup{Wtorek, 13:15}

\author{%
    \studentinfo[239661@edu.p.lodz.pl]{Szymon Gruda}{239661}
}

\title{Zadanie 1.: Problem Set 1}

\begin{document}
    \maketitle
    \thispagestyle{fancyplain}

    \section{Wprowadzenie} {
        Dane wraz ze wykorzystującą je statystyką pozwalają opisywać otaczający ludzi świat i informować co się w nim dzieje. Bardzo ważnym aspektem jest ich prezentacja (wizualizacja). Dużo trudniej jest ludziom zinterpretować długą tabelę z samymi liczbami, lepszą metodą jest zwizualizowanie danych, np. poprzez wykres. Niestety wizualizowanie danych jest podatne na różnego rodzaju manipulacje, które zostaną omówione poniżej.
    }

    \section{Przykłady manipulacji danymi podczas ich wizualizacji} {

    }

    \section{Wnioski} {
        Podsumowując wykonane zadanie wnioskuje, że:
        \begin{itemize}
            \item ...

        \end{itemize}
    }

    \begin{thebibliography}{0}
        \bibitem{l2short}{ }
        \bibitem{l2short}{ }
    \end{thebibliography}

\end{document}
